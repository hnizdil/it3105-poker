\documentclass[%
%a5paper,							% alle weiteren Papierformat einstellbar
%landscape,						% Querformat
12pt,								% Schriftgr��e (12pt, 11pt (Standard))
%BCOR1cm,							% Bindekorrektur, bspw. 1 cm
%DIVcalc,							% f�hrt die Satzspiegelberechnung neu aus
%											  s. scrguide 2.4
%twoside,							% Doppelseiten
%twocolumn,						% zweispaltiger Satz
%halfparskip*,				% Absatzformatierung s. scrguide 3.1
%headsepline,					% Trennline zum Seitenkopf	
%footsepline,					% Trennline zum Seitenfu�
titlepage						% Titelei auf eigener Seite
%normalheadings,			% �berschriften etwas kleiner (smallheadings)
%idxtotoc,						% Index im Inhaltsverzeichnis
%liststotoc,					% Abb.- und Tab.verzeichnis im Inhalt
%bibtotoc,						% Literaturverzeichnis im Inhalt
%abstracton,					% �berschrift �ber der Zusammenfassung an	
%leqno,   						% Nummerierung von Gleichungen links
%fleqn,								% Ausgabe von Gleichungen linksb�ndig
%draft								% �berlangen Zeilen in Ausgabe gekennzeichnet
]
{scrartcl}

%\pagestyle{empty}	
%\pagestyle{headings}
\usepackage[english]{babel}
\usepackage[T1]{fontenc}
\usepackage[ansinew]{inputenc}

%\usepackage{lmodern}

\usepackage{graphicx} 
\usepackage{booktabs} 
%\usepackage{subfig} 
\usepackage{color}
\usepackage{listings}

\begin{document}
\lstset{		language=Java,
				basicstyle=\ttfamily,
				showstringspaces=false,
				commentstyle=\itshape\color[rgb]{0.2,0.6,0.2},
				%numbers=left,
				keywordstyle=\bfseries\color[rgb]{0.6,0.0,0.4},
				tabsize=4
				}
\pagestyle{empty}

%\titlehead{Titelkopf}
\subject{Report}
\title{AI Poker Players}
\subtitle{Course IT3105}
\author{Robert Braunschweig \and Jan Hn\'{i}zdil}
%\and{Der Name des Co-Autoren}
%\thanks{Fu�note}		
%\date{}						
%\publishers{Herausgeber}

\maketitle 


%\begin{abstract}


%\end{abstract}
\tableofcontents		
%\listoftables			
%\listoffigures				
\newpage

\section{Introduction}

\section{Basic Approach}

\section{The Game Simulator}
\subsection{Concept}
\subsection{Implementation}

\section{AI Poker Player}

\subsection{Phase I Player}
\subsubsection{Structure}

We call phase I player BadBot and as every player in our game, he extends the
\texttt{Player} class. BadBot has only one method, which decides on action to take. He
differentiates between pre-flop and post-flop phases. In pre-flop phase, the bot
just calls. Unlike GoodBot (discussed later), bad one doesn't utilize any
pre-flop rollout simulation data.

After the flop cards have come out, bot can calculate its hand power, which is
done by the \texttt{calcCardsPower()} method in \texttt{Card} class. This method
finds the strongest hand in the supplied cards and returns array of integers as
proposed in the paper. The first number in the array represents the type of hand
-- one for high card, two for one pair up to nine for a straight flush. If the
bot has weaker hand than one pair (first number is less than two) he folds. On
the other hand, when having at least straight (first number bigger than four),
bot raises to the random amount which is between his own bet and maximal round
bet. He's allowed to raise only once per round, so he just calls if raised
already. In the remaining cases bot calls to stay in the game. Also when he
stays alone in the game, call occur so he can have the pot.

In conclusion our BadBot doesn't do anything fancy and only tries to obey the
basic rules of Texas Hold 'Em poker game.

\subsubsection{Testing}

We did basic testing of bots by letting them play. \texttt{Game} class takes
care of running the game by dealing the cards, running game rounds and splitting
the pot at the end. Game queries the active players about their actions during
round and decides on winner at the end of the final round.

Game can take two parameters. First set how many games should players play and
the second one is the name of the file with the initial commands. Those commands
set up the game at the beginning which involves number of players, initial
budget of players, max bet per round, etc. Game simulation is described better
in \textit{The Game Simulator} section.

As proposed in the project paper, we ran five different 1000-hand runs of game
eahc involving four BadBot players. Command used to run one 1000-hand run looks
like \texttt{java Game 1000 game-p1.txt}, where \texttt{game-p1.txt} is the file
containing game initialization commands. Results of these runs are summarized in
following five tables. Initial budget of each player was 10\,000 money with 500
being max bet per round.

\begin{center}

\vspace{5pt}
\begin{tabular}{lrr}
\toprule
\multicolumn{3}{c}{First 1000-hand run} \\
Player & Wins & Final budget \\
\midrule
BAD1 & 140 &  $-9\,922$ \\
BAD2 & 149 &  $91\,520$ \\
BAD3 & 126 & $-71\,487$ \\
BAD4 & 149 &  $29\,889$ \\
\bottomrule
\end{tabular}

\vspace{5pt}
\begin{tabular}{lrr}
\toprule
\multicolumn{3}{c}{Second 1000-hand run} \\
Player & Wins & Final budget \\
\midrule
BAD1 & 144 & $25\,910$ \\
BAD2 & 154 & $-61\,446$ \\
BAD3 & 147 & $-48\,897$ \\
BAD4 & 166 & $124\,425$ \\
\bottomrule
\end{tabular}

\vspace{5pt}
\begin{tabular}{lrr}
\toprule
\multicolumn{3}{c}{Third 1000-hand run} \\
Player & Wins & Final budget \\
\midrule
BAD1 & 147 & $143\,959$ \\
BAD2 & 155 & $182\,682$ \\
BAD3 & 142 & $-82\,924$ \\
BAD4 & 111 & $-203\,721$ \\
\bottomrule
\end{tabular}

\vspace{5pt}
\begin{tabular}{lrr}
\toprule
\multicolumn{3}{c}{Fourth 1000-hand run} \\
Player & Wins & Final budget \\
\midrule
BAD1 & 157 & $-38\,491$ \\
BAD2 & 139 & $35\,140$ \\
BAD3 & 140 & $43\,494$ \\
BAD4 & 143 & $-147$ \\
\bottomrule
\end{tabular}

\vspace{5pt}
\begin{tabular}{lrr}
\toprule
\multicolumn{3}{c}{Fifth 1000-hand run} \\
Player & Wins & Final budget \\
\midrule
BAD1 & 141 & $30\,258$ \\
BAD2 & 132 & $36\,669$ \\
BAD3 & 155 & $28\,031$ \\
BAD4 & 134 & $-54\,965$ \\
\bottomrule
\end{tabular}

\end{center}

%TODO more tables


\subsection{Phase II Player}

\subsubsection{Pre-Flop Rollout Simulation}

This simulation estimates the chances of hole card pair winning the whole game.
It's done before the flop cards are put in. We know only two hole cards and set
other players cards and community cards randomly. Now we're able to compare our
hand with other hands and see how it performs. This is just one round and since
it's based on randomness, we should run as many rounds as we can.

We could also calculate hand winning chance by concerning all possible
combinations of final card setups, but the number of these combinations is so
large, that we can't accomplish it in reasonable computation time. That's why we
do the simulation.

It's possible to decrease number of hole pairs we need to simulate by using
equivalence classes as described in the paper. Equivalence classes are of three
types:

\begin{enumerate}
\item unpaired and unsuited -- 78 combinations;
\item unpaired and suited -- 78 combinations;
\item pairs with the same value -- 13 combinations.
\end{enumerate}

That gives us 169 possible equivalence classes together. For each class, we run
at least 1000 rollout simulations recording number of wins, ties and losses.
From those numbers it's easy to calculate winning probability of the equivalence
class (and the hole card pair that's in the class). It's also essential to
differentiate probabilities by the number of players at the table. Having data
like this, player finds his hole card pair in the table entry (with the number
of players matching) and gets the winning probability.

For calculating the rollout simulations we used the cluster machine available to
NTNU students who have parallel programming course. Using MPJ (Java wrapper to
the MPI -- Message Passing Interface) we were able to run simulations parallely
on eight processors. Each processor got even part of the 169-item array of
equivalence classes and then ran simulations for each pair and number of players
going from two to ten. We managed to run almost five million runs for each
equivalence class, which gives us a little bit more then 500\,000 rollouts for
each class and each number of players.

Results of the rollouts were stored in the MySQL database. Then simple SQL query
was run to calculate probabilities per each class and each number of players.
Results of this query were stored into textfile approximately 40\,kB in size.
This file (\texttt{PreFlopTable.dat}) is used by the \texttt{PreFlopTable}
class, which pulls the probabilities out and stores them into internal Java
data structure. Our GoodBot player can then easily use the
\texttt{getWinningProbability()} method of the \texttt{PreFlopTable} class to
get winning probability for his hole pair. The method takes two arguments --
hole cards and number of players in the game.

\subsubsection{Hand-Strength Calculation}


\subsubsection{Testing}

Given above techniques GoodBot player can do better decisions on which actions
to take. In the pre-flop phase he uses the hole pair winning probabilities to
decide on folding, calling or even raising. Here comes a little tricky part
which is setting the probability thresholds. If the winning probability is less
than 0.085 our bot folds. Between 0.085 and 0.2 he calls and with higher
probability he raises random amount.

In the post-flop phases our bot has more information and hopefully can do better
decisions. He uses hand strength calculated according to the above subsection.
Threshold for those phases are 0.2 and 0.5. When raising, GoodBot takes into
account his hand strength.

As paper proposed, we again ran five 1000-hand game runs involving four BadBot
and four GoodBot players.

\begin{center}

\vspace{5pt}
\begin{tabular}{lrr}
\toprule
\multicolumn{3}{c}{First 1000-hand run} \\
Player & Wins & Final budget \\
\midrule
BAD1&133&-196300\\
BAD2&115&-598414\\
BAD3&132&-1077654\\
BAD4&122&-1569593\\
GOOD1&53&850387\\
GOOD2&55&752577\\
GOOD3&67&1155566\\
GOOD4&69&763426\\
\bottomrule
\end{tabular}

\vspace{5pt}
\begin{tabular}{lrr}
\toprule
\multicolumn{3}{c}{Second 1000-hand run} \\
Player & Wins & Final budget \\
\midrule
BAD1&119&-496395\\
BAD2&133&-409540\\
BAD3&107&-1200254\\
BAD4&135&-1457898\\
GOOD1&61&910192\\
GOOD2&54&872461\\
GOOD3&58&716223\\
GOOD4&68&1145209\\
\bottomrule
\end{tabular}

\vspace{5pt}
\begin{tabular}{lrr}
\toprule
\multicolumn{3}{c}{Third 1000-hand run} \\
Player & Wins & Final budget \\
\midrule
BAD1&129&-858307\\
BAD2&124&-719593\\
BAD3&94&-1343100\\
BAD4&146&-1106336\\
GOOD1&66&1354768\\
GOOD2&53&611992\\
GOOD3&59&1118677\\
GOOD4&65&1021894\\
\bottomrule
\end{tabular}

\vspace{5pt}
\begin{tabular}{lrr}
\toprule
\multicolumn{3}{c}{Fourth 1000-hand run} \\
Player & Wins & Final budget \\
\midrule
BAD1&129&-1006496\\
BAD2&120&-899251\\
BAD3&108&-654700\\
BAD4&163&-551934\\
GOOD1&60&1031790\\
GOOD2&48&409538\\
GOOD3&63&958067\\
GOOD4&63&792984\\
\bottomrule
\end{tabular}

\vspace{5pt}
\begin{tabular}{lrr}
\toprule
\multicolumn{3}{c}{Fifth 1000-hand run} \\
Player & Wins & Final budget \\
\midrule
BAD1&130&-910942\\
BAD2&112&-663131\\
BAD3&112&-1655700\\
BAD4&147&-1299193\\
GOOD1&50&700803\\
GOOD2&75&1764724\\
GOOD3&59&1198982\\
GOOD4&61&944450\\
\bottomrule
\end{tabular}

\end{center}


\section{Problems}

\section{Conclusion}


\end{document}
